%% LyX 2.3.0 created this file.  For more info, see http://www.lyx.org/.
%% Do not edit unless you really know what you are doing.
\documentclass[twocolumn,american]{article}
\usepackage[T1]{fontenc}
\usepackage[latin9]{inputenc}
\usepackage{geometry}
\geometry{verbose,lmargin=1.5cm,rmargin=1.5cm}
\usepackage{array}
\usepackage{units}
\usepackage{multirow}
\usepackage{amstext}
\usepackage{amssymb}
\usepackage{graphicx}

\makeatletter

%%%%%%%%%%%%%%%%%%%%%%%%%%%%%% LyX specific LaTeX commands.
%% Because html converters don't know tabularnewline
\providecommand{\tabularnewline}{\\}
%% A simple dot to overcome graphicx limitations
\newcommand{\lyxdot}{.}


\makeatother

\usepackage{babel}
\begin{document}

\title{2D Electron Transport}

\author{Alon Shaaltiel, Oren Kereth \& Gal Tuvia}
\maketitle
\begin{abstract}
By subjecting a hetero-junction of GaAs/AlGaAs to an external magnetic
field of varying strengths, at low temperature and under different
voltage gates, the quantum and classical Hall effects were observed.
Using the measurements, characteristics of the 2D electron gas (2DEG)
were studied and compared to values from the literature. The sample
was submerged in liquid helium under one atmosphere of pressure and
under a vacuum to achieve low temperatures of 4.2K and 1.7K, respectively.\global\long\def\tt#1{\times10^{#1}}
\end{abstract}

\part{Background}

In this experiment the Quantum Hall Effect (QHE) will be measured
for a 2D electron gas formed at the hetero-junction of AlGaAs and
GaAs.

\section{Classical Hall Effect}

At room temperature, when a small magnetic field ($B<1T)$ is applied
perpendicularly to a current running from one side of a metal plate
to the other, the classical Hall Effect can be observed and there
is a measurable potential difference between the two remaining sides
of the metal plate. This potential difference is normalized by the
current, yielding the Hall resistance \cite{StevenSimon}
\begin{equation}
R_{xy}=\text{\ensuremath{\frac{V_{H}}{I}}}\label{eq: Hall Resistance definition}
\end{equation}
where $R_{xy}$ is the xy component of the resistance tensor (and
also Hall resistance in this case)\cite{BADBOOKTHATISSOMEWHATUSEFULCLAUDE},
$V_{H}$ is the aforementioned potential difference and $I$ is the
current running through the metal plate. Using Drude's model one can
formulate the relation between $\rho_{xy}$, the resistivity, and
the applied magnetic field $B$ to be\cite{StevenSimon}
\begin{equation}
\rho_{xy}=\frac{1}{ne}B\label{eq: Classical Hall Effect main equation}
\end{equation}
where $n$ is the charge carrier density in the metal and $e$ is
charge of the electron. A similar expression can be written for the
resistance $R_{xy}$ due to the connection between the two. Note that
this is valid only for small enough magnetic fields, depending on
the temperature.

\section{Quantum Hall Effect}

QHE can be observed at low temperatures and for strong enough magnetic
fields ($B>1T$) for the 2D electron gas. As the name suggests, this
phenomena is explained using quantum mechanics. The 2D electron gas
is formed on the layer between two different semiconductors whose
Fermi level is different. Due to the difference in Fermi levels, near
the hetero-junction the band structure of both semiconductors is altered,
effectively creating a finite potential well for the electrons occupying
the boundary layer. As a result, these electrons are confined in their
movement and are practically two dimensional\cite{BADBOOKTHATISSOMEWHATUSEFULCLAUDE}.
Having a 2D electron gas is essential for the measurement of QHE as
it allows the creation of discrete energy states with large degeneracy
using a magnetic field known as Landau Levels\cite{BADBOOKTHATISSOMEWHATUSEFULCLAUDE}.
Measurement of $\rho_{xy},\rho_{xx}$ vs $B$ yields the following
graphs:
\begin{figure}
\includegraphics[scale=0.5]{\string"QHE GRAPHS - FROM THE BOOKS\string".jpg}

\caption{QHE - $\rho$, the resistivity in different directions vs $B$, the
applied magnetic field\cite{anactualusefulbookbySPRINGER} \label{fig:QHE- graphs from the book}}
\end{figure}
Note that for small magnetic fields $\rho_{xy}$ changes linearly
with $B$, fitting the classical Hall Effect, which is valid for small
magnetic fields as mentioned. However, for larger magnetic fields
plateaus of constant resistance at very specific values can be seen.
Moreover, corresponding to these plateaus on the $\rho_{xx}$ graph
there are also plateaus of 0 resistance (which are in fact infinite
resistance such that no current can flow). These graphs are analyzed
more thoroughly later on. Using the measurement of QHE, valuable information
regarding the electron gas can be extracted, as is done in the following
sections in this report.

\part{Experimental Setup}

As was previously mentioned, in this experiment we study the 2D electron
gas in the hetero-junction between AlGaAs and GaAs. In order to do
so, a sample of the aforementioned hetero-junction is lowered into
liquid helium with various probes, a temperature sensor and a capacitor.
The Dewar which holds the liquid helium also cooled a superconducting
coil which acts as a high-power magnet when current is passed through
it. The probes are drawn in Figure \ref{fig:Reperesentation of Sample},
probes 5 and 15 allow a current to be passed through the sample, probe
5 is connected to the source of a lock-in amplifier and probe 15 is
grounded. The same source is also connected in series to a $1\,\text{M}\Omega$
resistor, which is then grounded. As both probe 15 and the connection
to the resistor are grounded they are connected through the ``ground''
(a large metal board in our case). The resistance on the source is
$1\,\text{M}\Omega+R_{2DEG}(T)=R_{tot}$, as $R_{2DEG}(T)\approx1\,\text{k}\Omega$,
we can assume the resistance, and thus current, are not dependent
on the sample's temperature. This is done to reduce fluctuations in
later resistance measurements.

Probes 6 and 11 were used to measure the voltage across the 2D electron
gas with a lock-in amplifier, the lock-in amplifier then measures
the resistance across using the current that is fed from the source,
which is connected to probes 5 and 15. In a similar fashion, probes
6 and 3 were used to measure the resistance in perpendicular to the
current's direction.

Probe 1 is used to induce a voltage on the gold plate above the 2D
electron gas, the plate is grounded and like the resistor is thus
connected to probe 15. This allows it to carry a voltage potential
relative to the electron gas and create an electric field, which changes
the available states in the gas, thus changing the carrier density.

Both lock-in resistance readings, the temperature near the sample
and the magnetic field in the superconducting coil were fed into and
operated by a LabView application at the lab. The application could
command the parts to change one aspect (i.e. the gate voltage, or
current in the coil) at a given rate and measure all quantities during
the process. The working range was $-0.5$ Volt to $0.5$ Volt for
the gate voltage and $0$ to $5.5$ Tesla for the superconducting
coil. Do note that the temperature sensor's readings change under
a magnetic field.
\begin{figure}
\includegraphics[scale=0.1]{\string"Images/Sample Graph\string".png}

\caption{A representation of the hetero-junction sample, its probes and the
capacitor. The 2D electron gas is colored in dark gray, and the probes
and gold plate which serves as a capacitor, in orange. The capacitor
is positioned above the 2D electron gas. \label{fig:Reperesentation of Sample}}
\end{figure}
 In the first set of measurements, the resistance $R_{xx}$ was measured
versus the temperature $T$, changing from room temperature to $4.2K$
due to the boiling helium inside the Dewar. After the 2DEG stabilized
on $4.2K$ several measurements were made. One set of measurements
was done by setting the voltage gate to a constant value and then
measuring the resistances $R_{xx},R_{xy}$ while changing the magnetic
field from $0T$ to $5.5T$. The voltage gate value was either $0,-0.5$
or $0.5$ volts. Another set of measurements was made by setting the
magnetic field to a constant value of either $0.5T$ or $5T$ and
measuring the resistance while changing the voltage gate from $-0.5V$
to $0.5V$. After performing these measurements the temperature of
the sample was further lowered by creating a vacuum using a pump (see
Figure \ref{fig:Experimental-Setup DEWAR AND PUMP})
\begin{figure}
\includegraphics[scale=0.1]{\string"Images/Experimental Setup\string".jpg}\caption{Experimental Setup - in the middle: the Dewar which holds the helium.
Right above it the sample was inserted on a rod through the metal
pipe. To the right of the helium Dewar is the pump with which the
vacuum was created. \label{fig:Experimental-Setup DEWAR AND PUMP}}

\end{figure}
 After lowering the temperature to the vicinity of $1.7K$ (as there
were some fluctuations) the same measurements detailed above were
done again.

\part{Measurements}

\section*{Sheet Resistance vs Temperature}

As mentioned above, at the beginning of the experiment the sample
was cooled from 300K to 4.2K with $B=0$. During this cooling process,
its resistance $R_{xx}$ was measured and plotted versus the temperature.
Because the measured resistance is the total resistance of two resistors
connected in parallel - the first being the 2D electron gas and the
second being AlGaAs, effects regarding the resistance as a function
of temperature for both resistors must be taken into account. For
the 2D electron gas the resistance can be expressed as 
\begin{equation}
R_{2DEG}(T)=a_{0}+a_{1}T^{3}+a_{2}T^{5}\label{eq: Resistance as a function of Temp phonons}
\end{equation}
where $a_{0},a_{1},a_{2}$ are coefficients, $T$ is the temperature
and $R_{2DEG}$ is the resistance of the 2D electron gas. Note that
for $T\rightarrow0$ the resistance is not 0 due to impurities in
the material and that for higher temperatures the contributions of
$T^{3},T^{5}$ are due to phonons and thermal excitation occurring
in the electron gas\cite{StevenSimon}, leading to a higher resistance.
The behavior of the resistance as a function of the temperature for
the semiconductor is different than that of the 2DEG as thermal exications
in the semiconductor actually reduce the resistance by allowing the
electrons to move from the valence shell to the conductance shell,
thus contributing to the conduction of the material. Therefore the
resistance is
\begin{equation}
R_{SC}(T)=b_{0}+b_{1}e^{\frac{2E_{g}}{k_{B}T}}\label{eq: Resistance as a function of Temp SC}
\end{equation}
where $b_{0},b_{1},E_{g}$ are coefficients and $R_{SC}$ is the resistance
of the semiconductor. $2E_{g}$ is the energy gap between the valence
shell and the conductance shell  and it can be seen that for higher
temperatures the resistance becomes lower as opposed to the electron
gas whose resistance becomes higher. Because the two resistors are
parallel to each other the total resistance which is the one measured
is\cite{electricitystuffusedfromresistancesinparallel}
\begin{equation}
\frac{1}{R}=\frac{1}{R_{2DEG}}+\frac{1}{R_{SC}}\label{eq: PARALLEL RESISTORS, SC AND PHONON}
\end{equation}
Notice that for low temperatures $R_{SC}\rightarrow\infty$ and so
the semiconductor does not contribute to the total resistance and
for high temperatures the opposite occurs. Therefore, by fitting the
resistance in low temperatures to \ref{eq: Resistance as a function of Temp phonons}
and the resistance in high temperatures to \ref{eq: Resistance as a function of Temp SC}
all the coefficients $a_{0},a_{1},a_{2},b_{0},b_{1},E_{g}$ can be
extracted. . By measuring the resistance for each temperature and
by fitting the two temperature regions to the corresponding functions
the following graphs and parameters have been extracted:
\begin{figure}
\includegraphics[scale=0.5]{\string"Measurements/xlsx/1st Part/Changing Temperature Graphs/RVSTFullRange\string".jpg}\caption{Resistance VS Temperature - The entire range\label{fig:Resistance-VS-Temperature the entire range}}
\end{figure}
It can be seen that for low temperatures the resistance acts like
some polynomial and that for high temperatures it acts like an exponential
function, matching the theory above. By fitting each region the following
was extracted: 
\begin{figure}
\includegraphics[scale=0.5]{\string"Measurements/xlsx/1st Part/Changing Temperature Graphs/rvstsc_fitting\string".png}

\caption{Resistivity VS Temperature - High Temperature region\label{fig:Resistivity-VS-Temperature HIGH TEMP REGION}}
\end{figure}
 where $w$ and $L$ are the width and length of the sample correspondingly.
The corresponding parameters and statistical values.
\begin{table}
\begin{tabular}{|c|c|c|c|}
\hline 
Parameter & Value & Error & Relative Error\tabularnewline
\hline 
\hline 
$b_{0}[\Omega]$ & $72.010$ & $0.081$ & $0.11\%$\tabularnewline
\hline 
$b_{1}[\Omega]$ & $1.26\tt{-6}$ & $8.1\tt{-7}$ & $64\%$\tabularnewline
\hline 
$\unitfrac{E_{g}}{k_{B}}[K]$ & $1707$ & $67$ & $3.9\%$\tabularnewline
\hline 
$\chi_{red}^{2}$ & $1.3$ & \multicolumn{2}{c|}{------------}\tabularnewline
\hline 
$P_{value}$ & $0.0010$ & \multicolumn{2}{c|}{------------}\tabularnewline
\hline 
\end{tabular}

\caption{Resistivity VS Temperature Fit Parameters of the high temperature
region\label{tab:Resistance-VS-Temperature HIGH TEMP REGION}}
\end{table}
 For the low temperature region:
\begin{figure}
\includegraphics[scale=0.5]{\string"Measurements/xlsx/1st Part/Changing Temperature Graphs/rvstphonon_fitting\string".png}\caption{Resistivity VS Temperature - Low Temperature region\label{fig:Resistivity-VS-Temperature LOW TEMP REGION}}
\end{figure}
 with the corresponding parameters and statistical values
\begin{table}
\begin{tabular}{|c|c|c|c|}
\hline 
{\footnotesize{}Parameter} & {\footnotesize{}Value} & {\footnotesize{}Error} & {\footnotesize{}Relative Error}\tabularnewline
\hline 
\hline 
{\footnotesize{}$a_{0}[\Omega]$} & {\footnotesize{}$46.90$} & {\footnotesize{}$0.13$} & {\footnotesize{}$0.28\%$}\tabularnewline
\hline 
{\footnotesize{}$a_{1}[\unitfrac{\Omega}{K^{3}}]$} & {\footnotesize{}$1.0010\tt{-4}$} & {\footnotesize{}$9.2\tt{-7}$} & {\footnotesize{}$0.92\%$}\tabularnewline
\hline 
{\footnotesize{}$a_{2}[\unitfrac{\Omega}{K^{5}}]$} & {\footnotesize{}$3.134\tt{-9}$} & {\footnotesize{}$9.2\tt{-11}$} & {\footnotesize{}$2.9\%$}\tabularnewline
\hline 
{\footnotesize{}$\chi_{red}^{2}$} & {\footnotesize{}$0.28$} & \multicolumn{2}{c|}{{\footnotesize{}------------}}\tabularnewline
\hline 
{\footnotesize{}$P_{value}$} & {\footnotesize{}$1.0$} & \multicolumn{2}{c|}{{\footnotesize{}------------}}\tabularnewline
\hline 
\end{tabular}

\caption{Resistance VS Temperature Fit Parameters of the low temperature region\label{tab:Resistance-VS-Temperature LOW TEMP REGION}}
\end{table}
 For the high temperature region fit, the statistical values suggest
an under-estimation of error as the theory describes the behavior
of the resistance in that region properly, with the resistance changing
exponentially with $\frac{1}{T}$. The low temperature fit is adequate,
as can also be seen from the statistical values and from the fit itself.
Overall, both fits are good and the theory describes the physical
phenomena well. Using the fit parameters from both fits and equations
\ref{eq: Resistance as a function of Temp phonons},\ref{eq: Resistance as a function of Temp SC}
and \ref{eq: PARALLEL RESISTORS, SC AND PHONON} a graph consisting
of both the original measurements and the final fit for the entire
region was plotted (see Figure \ref{fig:1/R vs T for the entire region using all the measurements})
\begin{figure}
\includegraphics[scale=0.2]{\string"Measurements/xlsx/1st Part/TotalResistance_vs_Temp\string".png}\caption{$\frac{1}{R_{xx}}$ VS $T$ for the entire region of measurements.
In orange- equation \ref{eq: PARALLEL RESISTORS, SC AND PHONON} with
the fit parameters of both fits substituted in \ref{eq: Resistance as a function of Temp phonons}
and \ref{eq: Resistance as a function of Temp SC}. In blue- the measurements.
\label{fig:1/R vs T for the entire region using all the measurements}}
\end{figure}
 It can be seen that the final fitting function using the fit parameters
from the previous fits describes the physical phenomena adequately.
Further analysis can be found in the Discussion.

\section*{Low Field Regime}

As explained in the experimental setup, in this part of the experiment
the resistance of the 2DEG was measured for different gate voltages
and different constant temperatures. In low magnetic fields $R_{xy},R_{xx}$
exhibits a behavior described by the classical Hall Effect and the
Drude model \cite{StevenSimon}. By fitting a linear function 
\begin{equation}
R_{xy}=a_{0}+a_{1}B\label{eq: Classical Hall Effect linear fit to RXY}
\end{equation}
where according to \ref{eq: Classical Hall Effect main equation}
$a_{0}$ should be consistent with 0 and $a_{1}=\frac{1}{ne}$. Using
the linear fit, the charge carrier density of the 2DEG can be extracted
from $a_{1}$ by 
\begin{equation}
n=\frac{1}{a_{1}e}\label{eq: charge carrier density from classical hall effect}
\end{equation}
By measuring $R_{xx}$, which according to Drude model, is constant,
one can extract the mean free time between electrons, $\tau_{t}$,
through the relation between the two:
\begin{equation}
R_{xx}=\frac{m^{*}}{ne^{2}\tau_{t}}\label{eq: resistivity in drude model}
\end{equation}
where $m^{*}$ is the effective mass\cite{StevenSimon} of the 2DEG.
Several fits were made; 3 fits for each temperature (1.7K and 4.2K),
each fit with a different gate voltage $V_{G}$. By changing the gate
voltage the charge carrier density of the 2DEG changes, thus yielding
different linear graphs with varying slopes. Using the mean free time,
$\tau_{t}$, the mobility, $\mu$ ,of the 2DEG can be extracted \cite{StevenSimon}
\begin{equation}
\mu=\frac{e\tau_{t}}{m^{*}}\label{eq: mobility DRUDE MODEL}
\end{equation}
By fitting linear function to the 3 different gate voltages for $T=4.2K$
the following was extracted (see Figure \ref{fig:Classical-Hall-Effect R_xy vs B at T=00003D4.2K},
only $V_{g}=-0.5$ is shown, the rest are similar.) 
\begin{figure}
\includegraphics[scale=0.5]{\string"Measurements/xlsx/2nd Part/Classical Hall Effect/4.2K/linear_fittingminus05\string".png}\caption{Classical Hall Effect - $R_{xy}$ VS $B$ at $T=4.2K$ for $V_{g}=-0.5\,V$\label{fig:Classical-Hall-Effect R_xy vs B at T=00003D4.2K}}

\end{figure}
 with the corresponding $n$,$\tau_{t}$,$\mu$ and fit parameters
(see Tables \ref{tab:Fit-Parameters CLASSICAL HALL EFFECT T=00003D4.2K}
and \ref{tab:ExtractValues CLASSICAL HALL EFFECT T=00003D4.2K})
\begin{table}
\begin{tabular}{|c|c|c|c|}
\hline 
$V_{G}[V]$ & $n[10^{15}\unitfrac{1}{m^{2}}]$ & $\chi_{red}^{2}$ & $P_{value}$\tabularnewline
\hline 
\hline 
$0.5$ & $6.917\pm8.9\times10^{-2}(1.3\%)$ & $30$ & $0.0$\tabularnewline
\hline 
$0$ & $4.677\pm1.5\times10^{-2}(0.32\%)$ & $1.5$ & $0.0010$\tabularnewline
\hline 
$-0.5$ & $2.8803\pm5.0\times10^{-3}(0.17\%)$ & $0.52$ & $1.0$\tabularnewline
\hline 
\end{tabular}

\caption{Fit parameter extracted from the linear fits for $R_{xy}$ VS $B$
at $T=4.2K$\label{tab:Fit-Parameters CLASSICAL HALL EFFECT T=00003D4.2K}}

\end{table}
\begin{table}
\begin{tabular}{|c|c|c|}
\hline 
$V_{G}[V]$ & $\mu[\unitfrac{C*sec}{kg}]$ & $\tau_{t}[10^{-12}sec]$\tabularnewline
\hline 
\hline 
$0.5$ & $23.54\pm0.39(1.7\%)$ & $8.97\pm1.5\times10^{-1}(1.7\%)$\tabularnewline
\hline 
$0$ & $17.51\pm0.18(1.0\%)$ & $6.674\pm7.0\times10^{-2}(1.0\%)$\tabularnewline
\hline 
$-0.5$ & $8.961\pm0.092(1.0\%)$ & $3.413\pm3.5\times10^{-2}(1.0\%)$\tabularnewline
\hline 
\end{tabular}

\caption{Extracted values from the linear fits for $R_{xy}$ VS $B$ at $T=4.2K$\label{tab:ExtractValues CLASSICAL HALL EFFECT T=00003D4.2K}}
\end{table}
By fitting linear function to the 3 different gate voltages for $T=1.7K$
the following was extracted (see Figure \ref{fig:Classical-Hall-Effect R_xy vs B at T=00003D1.7K},
only $V_{g}=0.5V$ is shown while the rest are similar.)
\begin{figure}
\includegraphics[scale=0.5]{\string"Measurements/xlsx/2nd Part/Classical Hall Effect/1.7K/VG=0.5/linear_fitting\string".png}

\caption{Classical Hall Effect - $R_{xy}$ VS $B$ at $T=1.7K$ for $V_{G}=0.5V$\label{fig:Classical-Hall-Effect R_xy vs B at T=00003D1.7K}}
\end{figure}
 with the corresponding $n$,$\tau_{t}$,$\mu$ and fit parameters
(see Tables \ref{tab:Fit-Parameters CLASSICAL HALL EFFECT T=00003D1.7K}
and)
\begin{table}
\begin{tabular}{|c|c|c|c|}
\hline 
{\footnotesize{}$V_{G}[V]$} & {\footnotesize{}$n[10^{15}\unitfrac{1}{m^{2}}]$} & {\footnotesize{}$\chi_{red}^{2}$} & {\footnotesize{}$P_{value}$}\tabularnewline
\hline 
\hline 
{\footnotesize{}$0.5$} & {\footnotesize{}$6.693\pm4.0\times10^{-2}(0.60\%)$} & {\footnotesize{}$6.8$} & {\footnotesize{}$0.0$}\tabularnewline
\hline 
{\footnotesize{}$0$} & {\footnotesize{}$4.731\pm1.6\times10^{-2}(0.34\%)$} & {\footnotesize{}$2.0$} & {\footnotesize{}$0.0$}\tabularnewline
\hline 
{\footnotesize{}$-0.5$} & {\footnotesize{}$2.8173\pm4.9\times10^{-3}(0.17\%)$} & {\footnotesize{}$0.59$} & {\footnotesize{}$1.0$}\tabularnewline
\hline 
\end{tabular}

\caption{Fit Parameters from the linear fits for $R_{xy}$ VS $B$ at $T=1.7K$\label{tab:Fit-Parameters CLASSICAL HALL EFFECT T=00003D1.7K}}
\end{table}
 
\begin{table}
\begin{tabular}{|c|c|c|}
\hline 
{\footnotesize{}$V_{G}[V]$} & {\footnotesize{}$\tau_{t}[10^{-12}sec]$} & {\footnotesize{}$\mu[\unitfrac{C\cdot sec}{kg}]$}\tabularnewline
\hline 
\hline 
{\footnotesize{}$0.5$} & {\footnotesize{}$10.142\pm1.2\times10^{-2}(0.12\%)$} & {\footnotesize{}$26.623\pm0.032(0.12\%)$}\tabularnewline
\hline 
{\footnotesize{}$0$} & {\footnotesize{}$6.961\pm7.3\times10^{-2}(1.0\%)$} & {\footnotesize{}$18.27\pm0.19(1.0\%)$}\tabularnewline
\hline 
{\footnotesize{}$-0.5$} & {\footnotesize{}$3.338\pm3.4\times10^{-2}(1.0\%)$} & {\footnotesize{}$8.76\pm0.10(1.1\%)$}\tabularnewline
\hline 
\end{tabular}

\caption{Extracted values from the linear fits for $R_{xy}$ VS $B$ at $T=1.7K$}
\end{table}
Overall, for all the measurements at both temperatures and for all
the different values of the voltage gate it can be seen that the fits
are adequate and a linear fit indeed describes the phenomena (despite
the somewhat inadequate statistical values for some of them). In addition,
all values extracted from the fit parameters seem to be in line with
expected values. The charge carrier density came out lower than that
of other materials, fitting the information we have about the 2DEG
and what we were told in the lab. Moreover, $\tau_{t}$ is of the
appropriate scale compared to other materials\cite{Ashcroft}. Another
thing to note is that when lowering the temperature from $4.2K$ to
$1.7K$ it can be seen that $\tau_{t}$, on average, becomes longer.
Meaning there are less collisions in a given period of time. Further
analysis can be found in the Discussion.\global\long\def\e{\varepsilon}

As was mentioned in the 'Experimental Setup' section, measurements
under a constant magnetic field and a changing gate voltage were also
made. The relevant measurement to this section is that under a magnetic
field of $0.5\,Tesla$. The capacitance of the gate is $C=\frac{Q}{V_{g}}=\frac{Ane}{V_{g}}=\frac{\varepsilon_{0}\e_{r}A}{d}$
where the last relation is a known capacitor property. By dividing
both sides by $A$ we get $\frac{ne}{V_{g}}=\frac{\e_{0}\e_{r}}{d}$,
substituting $ne$ by $\frac{1}{R_{xy}}$, from the classical Hall
effect, and lastly, multiplying by $V_{g}$ we get the relation:

\begin{equation}
\frac{1}{R_{xy}}(V_{g})=\frac{\e_{0}\e_{r}}{d}V_{g}\label{eq:ResistanceDistance}
\end{equation}

where $\e_{0}\e_{r}$ is the permittivity of the sample (taken to
be the permittivity of $GaAs$), $d$ is the distance of the gate
from the sample, $V_{g}$ is the gate voltage and $R_{xy}$ is the
resistance perpendicular to current flow.

A linear fit of the form:

\[
y=a_{0}+a_{1}x
\]
 was made, where $y$ is $\frac{1}{R_{xy}}$, $x$ is $V_{g}$, $a_{1}$
has a theoretical value of $\frac{\e_{0}\e_{r}}{d}$ and $a_{0}$
has a theoretical value of $0$ (see Figure \ref{fig:distanceLinear}).

\begin{figure}
\includegraphics[scale=0.25]{\string"Measurements/xlsx/1st Part/Lin Fit/Fit\string".png}

\caption{A linear fit to $R_{xy}^{-1}$ as a function of $V_{g}$ under a magnetic
field of $0.5\,Tesla$ \label{fig:distanceLinear}}
\end{figure}

The fit parameters for the fit are presented in Table \ref{tab:DisanceFitParam}.

\begin{table}
\begin{tabular}{|c|c|c|c|}
\hline 
{\footnotesize{}Parameter} & {\footnotesize{}Value} & {\footnotesize{}Error} & {\footnotesize{}Relative Error}\tabularnewline
\hline 
\hline 
$a_{0}\,[\Omega^{-1}]$ & $1.4952\tt{-3}$ & $3.2\tt{-6}$ & $0.21\text{\%}$\tabularnewline
\hline 
$a_{1}\,[\unitfrac{\Omega^{-1}}{Volt}]$ & $1.061\tt{-3}$ & $1.0\times10^{-5}$ & $0.97\text{\%}$\tabularnewline
\hline 
$\chi_{red}^{2}$ & $4.3$ & \multicolumn{2}{c|}{{\footnotesize{}------------}}\tabularnewline
\hline 
$P_{value}$ & $4.4\tt{-42}$ & \multicolumn{2}{c|}{{\footnotesize{}------------}}\tabularnewline
\hline 
\end{tabular}

\caption{Fit parameters for the fit in Figure \ref{fig:distanceLinear}\label{tab:DisanceFitParam}}
\end{table}

The statistical values are out of their corresponding desired ranges,
yet Figure \ref{fig:distanceLinear} shows that the a linear fit describes
the phenomena adequately with some needed tweaks corresponding to
a higher order polynomial or an oscillating function. Using the permittivity
of $GaAs$ found in the literature \cite{doi:10.1063/1.363818} and
Eq. \ref{eq:ResistanceDistance} the distance of the gate from the
2D electron gas was found to be $d=1.097\tt{-7}\pm1.1\tt{-9}\,[m]$.

\section*{High Field Regime}

Continuing the last section, both $R_{xy}$ and $R_{xx}$ were measured
under a high magnetic field, up to 5.5 Tesla, while changing the gate
voltage on the 2D electron gas.

$R_{xy}$'s behavior is described by the Integer Quantum Hall Effect
and is characterized by plateaus along a linear line. The resistance
at each plateau is an integer multiple of $R_{K}=\frac{h}{e^{2}}=25800\,\Omega$\cite{BADBOOKTHATISSOMEWHATUSEFULCLAUDE}
, and because of spin degeneracy,  unless the spin states are separated,
only even integers are seen.

$R_{xx}$'s behavior is described by Shubnikov--de Haas oscillations,
which is characterized by big peaks corresponding to Landau Levels
in $R_{xy}$ and 0 resistances corresponding to the plateaus. 

If one looks at the behavior of $R_{xx}$ vs. $\frac{1}{B}$ all peaks
should come at a regular interval with decreasing amplitude. By calculating
that regular interval the carrier density can be found:

\begin{equation}
n=2\cdot\frac{e}{h\Delta(B^{-1})}\label{eq: finding the charge carrier density from SDH}
\end{equation}

where $h$ is Planck's constant, $e$ is the elementary charge , $\Delta(B^{-1})$
is the regular interval between peaks and the $2$ is due to the degeneracy
caused by spin\cite{BADBOOKTHATISSOMEWHATUSEFULCLAUDE}. 

The charge carrier density was calculated for all gate voltages and
both temperatures. The peaks for each configuration were found using
python and have a relative error of about 1\%.

Figure \ref{fig:Peaks in the resistance Rxx to find charge carrier density}
shows the identified peaks for the measurements under a gate voltage
of $V_{g}=0.5\,V$ and a temperature of $T=1.7\,K$.

\begin{figure}
\includegraphics[scale=0.35]{\string"Measurements/xlsx/2nd Part/Peaks/Peaks_Vg=0.5_T=1.7\string".png}

\caption{Plot of $R_{xx}$ as a function of $\frac{1}{B}$ for $V_{g}=0.5\,V$
and $T=1.7\,K$ with peaks highlighted by red x-es\label{fig:Peaks in the resistance Rxx to find charge carrier density}}
\end{figure}

As can be seen, the peaks are equidistant from each other with an
average distance of $\Delta(B^{-1})=0.06988\pm0.00027\,(0.38\%%)
\,\text{Tesla}^{-1}$.

The extracted carrier densities are presented in Table \ref{tab:Extracted charge carrier density from SDH- QHE}.

\begin{table}
\begin{tabular}{|c|c|c|}
\hline 
$V_{g}\,[V]$ & $T\,[K]$ & $n\,[\unitfrac{10^{15}}{m^{2}}]$\tabularnewline
\hline 
\hline 
$0.5$ & \multirow{3}{*}{$4.2$} & $6.921\pm0.026\,(0.38\%)$\tabularnewline
\cline{1-1} \cline{3-3} 
$0$ &  & $4.6875\pm0.0087\,(0.19\%)$\tabularnewline
\cline{1-1} \cline{3-3} 
$-0.5$ &  & $2.934\pm0.033\,(1.1\%)$\tabularnewline
\hline 
$0.5$ & \multirow{3}{*}{$1.7$} & $6.994\pm0.030\,(0.43\%)$\tabularnewline
\cline{1-1} \cline{3-3} 
$0$ &  & $5.012\pm0.080\,(1.6\%)$\tabularnewline
\cline{1-1} \cline{3-3} 
$-0.5$ &  & $3.235\pm0.085\,(2.6\%)$\tabularnewline
\hline 
\end{tabular}

\caption{Extracted carrier densities for the various configurations\label{tab:Extracted charge carrier density from SDH- QHE}}
\end{table}

These figures are in great agreement with those extracted using the
classic Hall effect. Notice that both final calculated densities have
a higher uncertainty, that is because one of the peaks was a peak
separated from others by the Zeeman effect corresponding to an odd
Landau Level (3 and 5).

Note how the amplitude of the peaks is lessened over smaller magnetic
fields (i.e bigger $\frac{1}{B}$s), the formula for the envelope
function of the SdH oscillations is \cite{PhysRevB.44.3793}: 

\begin{equation}
\Delta R(T,B^{-1})=4R_{0}\frac{\unitfrac{2\pi^{2}m^{*}k_{B}T\cdot B^{-1}}{\hbar e}}{\sinh(\unitfrac{2\pi^{2}m^{*}k_{B}T\cdot B^{-1}}{\hbar e})}e^{-\frac{\pi m^{*}}{e\tau_{q}}B^{-1}}\label{eq: Amplitude of oscilations SDH as a function of B and T}
\end{equation}
 where $\Delta R$ is the peak difference in resistance, $m^{*}$
is the electron's effective mass, $k_{B}$ is Boltzmann's constant,
$T$ is the 2DEG's temperature and $\tau_{q}$ is the scattering time
corresponding to the dephasing of the Landau state \cite{BADBOOKTHATISSOMEWHATUSEFULCLAUDE}.
Before fitting the data to the function written above the effective
mass, $m^{*}$ of the 2DEG has to be extracted. The resistance $R_{xx}$
was measured for two different temperatures - $T_{1}=1.7K$ and $T_{2}=4.2K$.
By dividing the amplitude of each peak measured at $T_{2}$ by its
corresponding peak at $T_{1}$ one can derive
\begin{equation}
y=\frac{T_{2}}{T_{1}}\frac{\sinh(\unitfrac{2\pi^{2}m^{*}k_{B}T_{1}B^{-1}}{\hbar e})}{\sinh(\unitfrac{2\pi^{2}m^{*}k_{B}T_{2}B^{-1}}{\hbar e})}e^{-\frac{\pi m^{*}}{e}(\frac{1}{\tau_{q}(T_{2})}-\frac{1}{\tau_{q}(T_{1})})B^{-1}}\label{eq: Ratio between amplitudes in SDH Oscilations to find m*}
\end{equation}
where the ratio was defined as $y=\frac{\Delta R(T_{2},B^{-1})}{\Delta R(T_{1},B^{-1})}$
and it was assumed that the scattering time $\tau_{q}$ is temperature
dependent. Therefore, by fitting 
\begin{equation}
y(x)=a_{0}\frac{sinh(\frac{T_{1}}{T_{2}}a_{1}x)}{sinh(a_{1}x)}e^{-a_{2}x}\label{eq: ratio fits SDH oscilations to find m*}
\end{equation}
where $x=\frac{1}{B}$ and the fact that both $\sinh$-s are the same
apart for the temperature was used, using the fit parameter $a_{1}$,
the effective mass can be found $m^{*}=\frac{a_{1}\cdot\hbar e}{2\pi^{2}k_{B}\cdot1.7}$
(see Figure \ref{fig:extracting the effective mass fit})
\begin{figure}
\includegraphics[scale=0.3]{\string"Measurements/xlsx/2nd Part/Factors/Vg=0\string".png}

\caption{$y$ VS $\frac{1}{B}$ for $V_{g}=0$\label{fig:extracting the effective mass fit}}

\end{figure}
The corresponding fit parameters and extracted effective mass are
(see Table...) 
\begin{table}

\begin{tabular}{|c|c|c|c|}
\hline 
{\footnotesize{}Parameter} & {\footnotesize{}Value} & {\footnotesize{}Error} & {\footnotesize{}Relative Error}\tabularnewline
\hline 
\hline 
$a_{0}\,[1]$ & $1.83$ & $0.98$ & $53\%$\tabularnewline
\hline 
$a_{1}\,[Tesla]$ & $1.9$ & $2.1$ & $110\%$\tabularnewline
\hline 
$a_{2}\,[Tesla]$ & $2.4$ & $3.5$ & $150\%$\tabularnewline
\hline 
$m^{*}\,[10^{-32}kg]$ & $6.8$ & $7.7$ & $110\%$\tabularnewline
\hline 
$\chi_{red}^{2}$ & $22$ & \multicolumn{2}{c|}{{\footnotesize{}------------}}\tabularnewline
\hline 
$P_{value}$ & $1.1\tt{-25}$ & \multicolumn{2}{c|}{{\footnotesize{}------------}}\tabularnewline
\hline 
\end{tabular}\caption{Fit parameters and extract effective mass from the $y$ VS $\frac{1}{B}$
at $V_{g}=0$ fit\label{tab:fit parameters of the fit from which the effective mass was extracted}}

\end{table}

After finding the effective mass from one of these fits, $\tau_{q}$
can now be extracted from the data using a fit of the form

\begin{equation}
\Delta R(T,x)=a_{0}\frac{a_{1}x}{\sinh(a_{1}x)}e^{-a_{2}x}\label{eq: fit for amplitude of oscilations SDH}
\end{equation}
was made to the peaks of each configuration after the average resistance
at a very low field was subtracted. Using the fitted parameters the
quantities $\tau_{q}$ can be calculated
\begin{equation}
\tau_{q}=\frac{\pi m^{*}}{ea_{2}}\label{eq: finding tau q from the SDH oscilations fit to the amplitude}
\end{equation}

Figure \ref{fig:R_xx as a function of B^-1 - SDH OSCILATIONS AMPLITUDE FIT FROM WHICH TAU was extracted}
shows one of the fits, which represents all others quite well in goodness
of fit.

\begin{figure}
\includegraphics[scale=0.35]{\string"Measurements/xlsx/2nd Part/Peaks/PeaksPlot_Vg=-0.5_T=1.7\string".png}

\caption{Plot of $R_{xx}$ as a function of $B^{-1}$ with resistance peaks
represented as red x-es. A fit to the envelope function was made and
is represented in orange\label{fig:R_xx as a function of B^-1 - SDH OSCILATIONS AMPLITUDE FIT FROM WHICH TAU was extracted}}
\end{figure}
The extracted $\tau_{q}$ values and their comparison with $\tau_{t}$
are shown in Table \ref{tab:tau_q values extracted from fits}
\begin{table}
{\footnotesize{}}%
\begin{tabular}{|c|c|c|c|c|c|}
\hline 
{\footnotesize{}$V_{g}\,[V]$} & {\footnotesize{}$T\,[K]$} & {\footnotesize{}$\tau_{q}[10^{-12}s]$} & {\footnotesize{}$\frac{\tau_{t}}{\tau_{q}}$} & {\footnotesize{}$\chi_{red}^{2}$} & {\footnotesize{}$P_{value}$}\tabularnewline
\hline 
\hline 
{\footnotesize{}$0.5$} & \multirow{3}{*}{{\footnotesize{}$4.2$}} & {\footnotesize{}$1.19\pm0.20(17\%)$} & {\footnotesize{}$7.7$} & {\footnotesize{}$1.6$} & {\footnotesize{}$0.099$}\tabularnewline
\cline{1-1} \cline{3-6} 
{\footnotesize{}$0$} &  & {\footnotesize{}$0.545\pm0.026(4.8\%)$} & {\footnotesize{}$12$} & {\footnotesize{}$1.0$} & {\footnotesize{}$0.14$}\tabularnewline
\cline{1-1} \cline{3-6} 
{\footnotesize{}$-0.5$} &  & {\footnotesize{}$1.04\pm0.17(16\%)$} & {\footnotesize{}$3.3$} & {\footnotesize{}$0.61$} & {\footnotesize{}$0.68$}\tabularnewline
\hline 
{\footnotesize{}$0.5$} & \multirow{3}{*}{{\footnotesize{}$1.7$}} & {\footnotesize{}$0.3684\pm0.0094(2.6\%)$} & {\footnotesize{}$28$} & {\footnotesize{}$0.68$} & {\footnotesize{}$0.61$}\tabularnewline
\cline{1-1} \cline{3-6} 
{\footnotesize{}$0$} &  & {\footnotesize{}$0.448\text{\ensuremath{\pm0.016}}(3.6\%)$} & {\footnotesize{}$16$} & {\footnotesize{}$0.35$} & {\footnotesize{}$0.98$}\tabularnewline
\cline{1-1} \cline{3-6} 
{\footnotesize{}$-0.5$} &  & {\footnotesize{}$0.4156\pm0.0036(0.87\%)$} & {\footnotesize{}$8.3$} & {\footnotesize{}$0.40$} & {\footnotesize{}$0.94$}\tabularnewline
\hline 
\end{tabular}\caption{$\tau_{q}$ extracted from SDH oscillations fit of $R_{xx}$ VS $\frac{1}{B}$\label{tab:tau_q values extracted from fits}}

\end{table}
 On average $\tau_{q}$ is smaller than $\tau_{t}$ in about one order
of magnitude. Further analysis is presented in the Discussion.

\section{Quantum Hall Regime}

In this part of the experiment the plateaus in the $R_{xy}$ graph
(see Figure \ref{fig:QHE- graphs from the book}) were extracted and
their corresponding resistance value was used in order to extract
the filling factor $\nu$ using the equation \cite{BADBOOKTHATISSOMEWHATUSEFULCLAUDE}
\begin{equation}
R_{xy}=\frac{1}{\nu}\frac{h}{e^{2}}\label{eq: QHE Filling factor equation}
\end{equation}
where $h$ is the Planck constant. First of all the plateaus in each
set of measurements were identified (see Figures \ref{fig:Rxy vs B - QHE - Filling factor graph 4.2}
and \ref{fig:Rxy vs B - QHE - Filling factor graph 1.7}, other configurations
are similar.)
\begin{figure}
\includegraphics[scale=0.3]{\string"Measurements/xlsx/2nd Part/Plateaus/Plateau_Vg=0.5_T=4.2\string".png}\caption{$R_{xy}$VS $B$ with $T=4.2K$ and $V_{G}=0.5V$. The plateaus are
labeled in red. \label{fig:Rxy vs B - QHE - Filling factor graph 4.2}}
\end{figure}

\begin{figure}
\includegraphics[scale=0.3]{\string"Measurements/xlsx/2nd Part/Plateaus/Plateau_Vg=0.5_T=1.7\string".png}\caption{$R_{xy}$VS $B$ with $T=1.7K$ and $V_{G}=0.5V$ . The plateaus are
labeled in red. \label{fig:Rxy vs B - QHE - Filling factor graph 1.7}}
\end{figure}
The filling factors extracted for each graph can be seen in Table
\ref{tab:Filling-Factor Table for every temp and voltage gate}
\begin{table}
\begin{tabular}{|c|c|c|}
\hline 
\multicolumn{3}{|c|}{{\footnotesize{}Filling Factors Table}}\tabularnewline
\hline 
\hline 
{\footnotesize{}$T[K]$} & {\footnotesize{}$V_{G}[V]$} & {\footnotesize{}First Extracted $\nu$-s}\tabularnewline
\hline 
\multirow{3}{*}{{\footnotesize{}$4.2$}} & {\footnotesize{}$-0.5$} & {\footnotesize{}$8.10,6.08,4.08,2.09$}\tabularnewline
\cline{2-3} 
 & {\footnotesize{}$0$} & {\footnotesize{}$10.12,8.10,6.07,4.06$}\tabularnewline
\cline{2-3} 
 & {\footnotesize{}$0.5$} & {\footnotesize{}$12.13,10.14,8.13,6.11$}\tabularnewline
\hline 
\multirow{3}{*}{{\footnotesize{}$1.7$}} & {\footnotesize{}$-0.5$} & {\footnotesize{}$6.08,4.07,3.05,2.06$}\tabularnewline
\cline{2-3} 
 & {\footnotesize{}$0$} & {\footnotesize{}$8.09,6.07,5.08,4.06$}\tabularnewline
\cline{2-3} 
 & {\footnotesize{}$0.5$} & {\footnotesize{}$12.13,10.14,8.13,6.11$}\tabularnewline
\hline 
\end{tabular}

\caption{Filling Factor $\nu$ for each temperature and voltage gate\label{tab:Filling-Factor Table for every temp and voltage gate}}

\end{table}
For all graphs the filling factor $\nu$ came out quite close to an
integer value. For $T=4.2K$, $\nu$ is even, as was expected due
to spin degeneracy and as was explained in previous sections. However,
for $T=1.7K$, $\nu$ took on odd values as well. This is due to the
spin degeneracy being lifted as a result of the spin interaction with
the magnetic field- Zeeman effect \cite{SAKURAI}. Further analysis
can be found in the discussion. After extracting the filling factors
both $R_{xx}$and $R_{xy}$ were plotted VS the voltage gate $V_{g}$
at a constant magnetic field of $B=5T$ yielding (see Figures \ref{fig:-vs-RXX5Tesla}
and \ref{fig:-vs-RXY5Tesla})
\begin{figure}
\includegraphics[scale=0.3]{\string"Measurements/xlsx/3rd Part/RxxAt5Tesla\string".png}

\caption{$R_{xx}$ vs $V_{g}$ at a magnetic field of 5 Tesla for both temperatures.
Orange- 4.2K and blue- 1.7K\label{fig:-vs-RXX5Tesla}}
\end{figure}
\begin{figure}
\includegraphics[scale=0.3]{\string"Measurements/xlsx/3rd Part/RxyAt5Tesla\string".png}

\caption{$R_{xy}$ vs $V_{g}$ at a magnetic field of 5 Tesla for both temperatures.
Orange- 4.2K and blue- 1.7K\label{fig:-vs-RXY5Tesla}}
\end{figure}
For both temperatures the change in voltage gate led to a change in
the highest Landau level being occupied. This can be identified in
$R_{xy}$ by noticing the plateau at around $V_{g}=-0.1V$ for $T=1.7K$
or around $V_{g}=0.1V$ for $T=4.2K$. A part of SDH oscillations
can also be spotted in $R_{xx}-$at corresponding values of the voltage
gate a sudden drop in resistance can be noticed. Further analysis
at the Discussion.

\part{Discussion}

\section*{Sheet Resistance VS Temperature}

In this part of the experiment the sheet resistance $R_{xx}$ was
fitted according to the model of two resistors connected in parallel,
one is the 2DEG and the other is a semiconductor. Each of these resistors
behaves differently under temperature change and become dominant in
different temperature regions. We used this information in order to
perform 2 fits- one in high temperatures and the other in low temperatures.
Using the fit parameters extracted from these two fits another plot
was made for the entire region using \ref{eq: PARALLEL RESISTORS, SC AND PHONON}
yielding Figure \ref{fig:1/R vs T for the entire region using all the measurements}.
As was mentioned previously it can be seen that the model describes
the phenomena adequately. It can be seen, however, that in the middle
of the graph (around $T=160K$) there is somewhat of an inconsistency
between the function and the original measurements. The functions
seems somewhat ``delayed'' compared with the measurements. A source
for this problem could be that the two previously mentioned fits,
each in its respective regions, were still effected by the existence
of the other region; i.e. in low temperatures, towards temperatures
of about $T\sim100K$, the semiconductor would still have an effect
on the fit that was not taken into account. The same thing can be
said about higher temperatures of about $T\sim200K$ where the 2DEG
affects the fit for the semiconductor. This explanation is also validated
by the two fits \ref{fig:Resistivity-VS-Temperature HIGH TEMP REGION}
and \ref{fig:Resistivity-VS-Temperature LOW TEMP REGION} where towards
the middle range temperature (between $100-200K)$ the measurements
in both fits seem to deviate from the fitted function, also leading
to the somewhat inadequate statistical values extracted from these
fits. Regardless, the model we used in this part of the experiment
seems to work well in describing the change of the resistance with
temperature.

\section*{Low Field Regime}

Using the classical Hall Effect in equation \ref{eq: Classical Hall Effect main equation},
Drude model and by measuring $R_{xx},R_{xy}$ in low magnetic fields,
the charge carrier density, $n$, the free time between collision,
$\tau_{t}$ and the mobility of the charge carrier, $\mu$, were all
extracted at different temperatures and voltage gates. In all graphs
it can be seen that the linear fit describes the dependency of the
resistance in the magnetic field for low fields quite well. The good
fit with lacking statistical parameters suggest an under-estimation
of uncertainty. Regardless, the fits were adequate and the extracted
values are in line with what we expected. The charge carrier density
changes with the voltage gate in an expected manner- higher voltage
gate leads to higher charge carrier density (and not the opposite
- since the electrons have negative charge).\\
In the second part of the Low Field Regime section we tried to extract
the distance of the gate from the sample. The statistical parameters
were outside their desired ranges, as the fit is overall very good,
this is likely because of the very small uncertainty in the measurements.
The extracted distance is of the a scale likely to be used in this
experiment. As we do not know the real distance, no comparison can
be made.

\section*{High Field Regime}

In this section of the experiment we analyzed the behavior of the
2D electron gas under a high magnetic field ($B>0.5\,Tesla$). We
observed Landau Levels, the integer quantum Hall effect and Shubnikov--de
Haas oscillations. We started by extracting the carrier densities
for different temperatures and gate voltages. These can be compared
to the carrier densities extracted using the classical Hall effect
and are at a distance of $1\%$ to $10\%$ at worst. Do also note
that the carrier density is directly correlated with gate voltage,
the higher the gate voltage the higher the carrier density, just as
it was in the low field regime. This is in great agreement with the
theory. Afterwards, using the peaks of the oscillations and the known
envelope function we attempted to extract the effective mass and quantum
lifetime. The effective mass of the 2DEG came out close to the theoretical
value however, it has a large uncertainty. Using the effective mass
and by performing several fits, the quantum lifetime was extracted
for different temperatures and voltage gates. The fits from which
$\tau_{q}$ was extracted were quite good, yielding good statistical
values. Comparing the extracted values of $\tau_{q}$ to those in
the literature \cite{QuantumLifetime} yielded good results as they
are of the same magnitude. However, it is clear there is some difference
between the values, this can also be demonstrated in the ratio $\unitfrac{\tau_{t}}{\tau_{q}}$
which came out smaller compared to \cite{QuantumLifetime}. An explanation
for these inconsistencies could be that the temperature was not constant
throughout the measurements. When the measurements for $T=1.7K$ were
made, $T$ was not always constant, as a matter of fact it got lower
throughout our measurements in a way we could not control. The changing
temperature has a great effect on the fit from which the effective
mass was extracted as it was assumed that the two temperatures were
constant. If the temperatures changed between peaks, a fit that assumes
otherwise will not be as good. A possible explanation for the difference
in goodness of fit between the effective mass and the quantum lifetime
portion is the general behavior of the fitting functions. The function
$\frac{ax}{sinh(ax)}$ decays, a larger $a$ leads to a faster decay
rate. Multiplying this function by a decaying exponent yields $\frac{ax}{sinh(ax)}e^{-bx}$
(which is the function fitted to find $\tau_{q}$). Since these are
two decaying functions, both of which decay exponentially, over a
short enough sample, it can be quite well approximated as $Ae^{-Cx}$
where $C$ takes into consideration both decay rates. This practically
means that the first decaying function is quite redundant and that
it has little effect on how well the fit could be (as a matter of
fact we tried fitting the peaks of $R_{xx}$ to a decaying exponent
which led to just as good fits). As a result, the small temperature
changes would have almost no effect on the quality of the fit, but
they will alter the decay rates, that are directly connected to $\tau_{q}$.
Conversely, in order to find $m^{*}$ a fit of the form $\frac{sinh(ax)}{sinh(bx)}e^{-cx}$
was made. Unlike the first case, these are not two decaying functions
multiplied by each other and the part of $\frac{sinh(ax)}{sinh(bx)}$
could not be replaced by the exponent as before. As a result, the
changes in temperature had a genuine effect on that fit.

\section*{Quantum Hall Regime}

The first few filling factors in different graphs of $R_{xy}$ vs
$B$ for different temperatures and voltage gates were extracted.
The extracted values are quite close to integer values, fitting the
theory. In addition to that, they are almost all even, fitting the
spin degeneracy that disables odd filling factors. Only for $T=1.7K$,
where the Zeeman effect becomes more dominant, can the spin degeneracy
be lifted and odd filling factors are extracted. Some filling factors
are not as close to integral values as we would have wanted them to
be. The somewhat inconsistent values could be the result of inaccurate
measurement of the resistance by the lock-in amplifier. An additional
cause may be impurities and other types of scattering events not accounted
for which are amplified with increasing temperatures. Moving on to
the two graphs of $R_{xx}$ and $R_{xy}$ VS $V_{g}$; it was established
above that a change in the highest occupied Landau level occurs- for
$R_{xy}$ a plateau can be seen, and for $R_{xx}$ there is a clear
dip in the resistance (and at $T=1.7K$ the Zeeman effect can also
be noticed at the region of the dip in resistance). The reason why
the highest occupied Landau level changes is because the Fermi level
changes. As shown in previous sections, the charge carrier density
becomes greater for higher voltage gates. The Fermi level also becomes
higher when the charge carrier density becomes greater, leading it
to eventually pass by a Landau level and cause the effect shown above.

\bibliographystyle{plain}
\bibliography{\string"2DEG report- bibtex\string"}

\end{document}
